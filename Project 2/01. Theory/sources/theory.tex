\section{Lý thuyết mô hình Markov ẩn}
Mô hình Markov ẩn (Hidden Markov model $-$ HMM) là một mô hình máy học cổ điển thông dụng trong việc xử lý chuỗi.
\subsection{Các thành phần của một mô hình Markov ẩn là gì? Chúng khác gì với mô hình Markov?}
Một mô hình Markov ẩn có cấu tạo như sau \supercite{angrybird, hmm2021}:
\begin{mybox}
\begin{itemize}
\item $Q = {q_1}{q_2} \ldots {q_N}:$ tập hợp $N$ trạng thái (states).
\item $A = {a_{11}} \ldots {a_{ij}} \ldots {a_{NN}}:$ ma trận xác suất chuyển (transition probability matrix) $A,$ $a_{ij}$ là xác suất chuyển từ trạng thái $i$ sang trạng thái $j,$ $\sum\limits_{j = 1}^N {{a_{ij}}}  = 1,\forall i.$
\item $O = {o_1}{o_2} \ldots {o_T}:$ chuỗi các quan sát (observations), lấy từ bộ từ vựng (vocabulary) $V = {v_1}, {v_2}, \ldots, {v_V}.$
\item $B = b_i \left( {o_t} \right):$ ma trận xác suất phụ thuộc trạng thái (emission probabilities), thể hiện xác suất một quan sát $o_t$ được tạo thành từ trạng thái $i.$
\item $\pi = {\pi_1}, {\pi_2}, \ldots, {\pi_N}:$ phân phối xác suất ban đầu theo trạng thái, có nghĩa là $\pi_i$ thể hiện xác suất xích Markov bắt đầu ở trạng thái $i.$ Một số trạng thái $j$ có thể có $\pi_j = 0,$ do chúng không thể là trạng thái ban đầu của xích Markov.
\end{itemize}
\end{mybox}
So với xích Markov (Markov chains), mô hình Markov ẩn có thêm 2 thành phần là $O$ $-$ chuỗi các quan sát và $B$ $-$ ma trận xác suất phụ thuộc trạng thái.

\subsection{Các giả thiết (assumptions) đặt ra cho mô hình Markov ẩn là gì? Tìm ví dụ các bài taosn mà các giả thiết này hợp lý và bất hợp lý.}
\textbf{Các giả thiết (assumptions) đặt ra cho mô hình Markov ẩn.}\\
Một mô hình Markov ẩn có 2 giả thiết chính:
\begin{itemize}
\item Xác suất của một trạng thái cụ thể chỉ phụ thuộc vào trạng thái ngay trước đó (Markov Assumptions):
\begin{equation}
P\left( {\left. {{q_i}} \right|{q_1} \ldots {q_{i - 1}}} \right) = P\left( {\left. {{q_i}} \right|{q_{i - 1}}} \right).
\end{equation}

\end{itemize}